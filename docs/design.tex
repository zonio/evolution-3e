\documentclass[a4paper,10pt]{article}
\usepackage{palatino}
\usepackage[utf8]{inputenc}
\usepackage[czech]{babel}
\usepackage[svgnames]{xcolor}
\usepackage{vmargin,calc,float,epsfig,multicol,fancyheadings,enumitem,lineno}
\GetGinDriver
\usepackage[\GinDriver,a4paper,breaklinks,colorlinks,pdfpagemode=None,pdfstartview=Fit,pdfpagelayout=SinglePage]{hyperref}

% set some basic values
\setlength{\columnseprule}{0.25mm}
\setlength{\columnsep}{1cm}
\setlength{\parindent}{0pt}
\setlength{\parskip}{2.5mm}
\setcounter{tocdepth}{2}

% customize environments
\setdescription{itemsep=0pt,style=nextline,leftmargin=10pt}%inline
\setitemize{itemsep=0pt,leftmargin=20pt}
\setenumerate{itemsep=0pt,leftmargin=20pt}

% pageformat
\normalsize
\setpapersize{A4}
\setmarginsrb{10mm}{10mm}{10mm}{10mm}{2\baselineskip}{\baselineskip}{\baselineskip}{2\baselineskip}
\lhead{\large\doccolor\textbf{\docsubtitle}}
\rhead{\doccolor\textsc{\textbf{\doctitle}}}
\lfoot{\doccolor\textbf{Strana \thepage}}
\rfoot{\doccolor\textbf{\today}}
\cfoot{\doccolor\textbf{\docauthors}}
\setlength{\footrulewidth}{0.5mm}
\setlength{\headrulewidth}{0.5mm}
\pagestyle{fancy}

%%
%% Actual Document Follows:
%%

\newcommand{\doccolor}{\color[HTML]{be192a}}%zonio
\newcommand{\doctitle}{Zonio --- 3E}
%\newcommand{\doccolor}{\color[HTML]{4158a3}}%personal
%\newcommand{\doctitle}{Osobní}

\newcommand{\docsubtitle}{Návrh pluginů 3E do Evolution}
\newcommand{\docauthors}{Ondřej Jirman}

\begin{document}
%\frontpage
%\tocpage

\begin{multicols*}{2}

\section{Přehled}

Cílem tohoto dokumentu je popsat vlastnosti a návrh implementaci podpory pro
kalendářový systém 3E (Enterprise Event Exchange?\footnote{Nebo jak to je?
Použít slovíčko \textit{Exchange} může být dobrý nápad, protože může k nasšemu systému
zavést lidi kteří budou vyhledávat MS Exchange ve vyhledávači. Což je ta správná
cílová skupina. :)})

Implementace podpory pro 3E kalendářový systém v evolution vyžaduje, díky
rozdělení evolution na dvě části, dva pluginy:
\begin{description}
\item[3E Evolution Plugin] Plugin implementující grafické uživatelské rozhraní.
\item[3E Evolution Data Server Plugin] Plugin implementjící manipulaci s obsahem
kalendáře a komunikaci se serverem při plánování meetingů.
\end{description}

Oba tyto pluginy jsou obsaženy v balíku \textit{evolution-3e}. Tento zdrojový
balík dále obsahuje popis komunikačního rozhraní 3E serveru. Balík vyžaduje pro
kompilaci knihovnu \textit{libxr} a nainstalovanou \textit{evolution} a
\textit{evolution-data-server}.

Pro správnou funkci pluginů je nezbytné aplikovat na \textit{evolution} a
\textit{evolution-data-server} patche uvedené v závěru. Tyto patche jsou
obsaženy v balíku \textit{evolution-3e}.

\section{Evolution plugin}

\subsection{Vlastnosti}

\begin{description}
\item[Automatické nastavení] Plugin projde seznam identit uživatele a pokusí se
zjistit zda pro dané domény existuje v DNS TXT záznam pro 3E server a zda
uživatel existuje na daném serveru. Pokud ano pak nabídne přihlášení. Po
úspěšném přihlášení uloží heslo do klíčenky pro bzudoucí použití. K aktivaci
pluginu dojde až po prvním přepnutí do kalendářové komponenty.
\item[Automatická aktualizace nastavení] Plugin periodicky kontaktuje 3E server
a synchronizuje seznam kalendářů a účtů v Evolution se změnami v seznamu kalendářů
na serveru.
\item[Offline režim] Plugin sleduje nastavení online/offline režimu v evolution
a povoluje/znemožňuje provádět činnosti které nelze provádět v offline režimu.
Mezi tyto činnosti patří jakákoliv manipulace se seznamem kalendářů, zaspisování
si sdílených kalendářů, změny práv, atd. Jediné co lze v offline režimu provádět
jsou manipulace s obsahem kalendáře. V offline režimu nemá prakticky smysl plánovat
meetingy, protože není možné přistupovat k free/busy informacím na serveru.
\item[Aktualizace nastavení] Plugin umožňuje aktualizaci nastavení kalendáře, 
jako je barva a uživatelský název kalendáře.
\item[Vytvoření nového kalendáře] Plugin umožňuje vytvořit nový kalendář pod
daným účtem kliknutím na ``New Calendar", následované výběrem účtu z seznamu a
zadáním názvu a barvy kalendáře. Název kalendáře se stane jeho identifikátorem,
který už později nepůjde změnit a bude se používat jako součást stabilního globálně
unikátního řetězece identifikujícího kalendář (URI). Název lze změnit, ale
identifikátor již zůstane neměnný.
\item[Odstranění kalendáře] Kalendáře vlastněné uživatelem lze odstranit pomocí
tlačítka ``Remove". Toto tlačítko není přístupné u nasdílených kalendářů.
\item[Zápis sdíleného kalendáře] Plugin umožňuje uživateli zapsat si cizí
kalendáře, pokud jejich vlastník povolil našemu uživateli k nim přístup. 
Zapsáním kalendáře se kalendář přidá do seznamu kalendářů uživatele. Zápis se
provádí pomocí položky ``Subscribe to..." v menu ``File". V dialogu který se
zobrazí musí uživatel vybrat kalendář který si chce zapsat. Uživatel může
vyhledat vlastníka kalendáře, který si chce zapsat pomocí vstupu s funkcí
automatického doplňování. Automaticky se vyhledává v seznamu uživatelů ze všech
kalendářových účtů které má uživatel v Evolution k dispozici.
\item[Zrušení zápisu sdíleného kalendáře] Plugin umožňuje zrušit zápis kliknutím
pravým tlačítkem myši na zapsaný kalendář v seznamu kalendářů a výběrem položky
``Unsubscribe" z menu.
\item[Modifikace přístupových práv] Plugin umožňuje uživateli nastavovat práva
na jeho vlastním kalendáři kliknutím pravým tlačítkem myši na kalendář v seznamu
kalendářů a výběrem položky ``Set permissions..." z menu. Následně se zobrazí
dialog nabízející tři režimy přístupu ke kalendáři:
  \begin{description}
  \item[Privátní kalendář] Kalendář bude přístupný pouze vlastníkovi kalendáře.
  \item[Veřejný kalendář] Kalendář bude přístupný pro čtení jakémukoliv
  přihlášenému uživateli.
  \item[Sdílený kalendář] kalendář bude přístupný vybraným uživatelům s
  zvolenými právy buď pro zápis nebo jen pro čtení.
  \end{description}
Pokud si uživatel vybere třetí možnost, bude mu umožněno upravit seznam
uživatelů kteří budou mít přístup k jeho kalendáři.
\end{description}

\subsection{Poznámky k implemetaci}

\subsubsection{Synchronizace seznamu kalendářů}

Seznam identit je v evolution reprezentován objktem EAccountList, což je seznam
objektů EAccount, které obsahují informace o identitách. Z objektu EAccountList
server určí seznam emailových adres reprezentujících, možné kalendářové účty,
které chce uživatel využívat. Tento sezname se dále zredukuje o adresy, k jejimž
doménovým jménům v DNS neexistuje TXT záznam pro 3E server.

Pro každý účet je seznam kalendářů uložen na serveru a je třeba jej načíst do 
evolution. Seznam kalendářů je v evolution reprezentován objektem ESourceList,
který obsahuje skupiny kalendářů reprezentované objekty ESourceGroup, které
obsahují samotné ESource objekty reprezentující jednotlivé kalendáře.
Seznam skupin a kalendářů je uložen v gconf a tyto objekty reagují na změny v
gconf tak že se automaticky synchronizují s novým stavem a upozorňují svého
uživatele pomocí signálu na to že byla provedena aktualizace.

Tj. lze si vytvořit instanci ESource objektu reprezentujícího určitý kalendář v
úplně separátním procesu od evolution a provést změny a tyto změny se okamžitě
projeví v běžící instanci evolution. Samozřejmě to funguje i v rámci jednoho
procesu čehož využívá 3E Evolution Plugin.

V rámci 3E plugin ESourceGroup reprezentuje uživatele 3E serveru a ESource
objekty v dané skupině reprezentují jeho kalendáře.

Cílem synchronizace tedy je z EAccountList získat seznam 3E účtů a pro tyto účty
získat seznam kalendářů ze serveru a z těchto informací aktualizovat současnou
hierarchii ESourceGroup a ESource objektů.

Nejen výše popsanou synchronizaci ale obecně manipulaci s 3E kalendáři má na
starosti objekt EeeAccountsManager, což je singleton vytvářený při inicializaci
pluginu.

EeeAccountsManager obsahuje seznam EeeAccount objektů, které reprezentují jak
účty našeho uživatele získané z identit v EAccountList, tak účty vlastníků
zapsaných kalendářů, ke kterým má uživatel přístup skrze své vlastní účty.

Toto rozdělení je zde proto, protože ESourceGroup objekty jsou přímo mapované
na EeeAccount objekty. To znamená, že zapsané kalendáře cizích uživatelů se
zobrazují jakoby pod cizím uživatelem v seznamu kalendářů.

EeeAccount objekty pak obsahují seznam EeeCalendar objektů, které jsou mapovány
na ESource objekty.

Samotná synchronizace pak probíhá tak, že se vytvoří hierarchie EeeAccount a
EeeCalendar objektů načtením ze serveru a pak se seznam EeeAccount objetů srovná
se seznamem ESourceGroup objektů vztažených k 3E pluginu a ESourceGroup objekty
které tam jsou navíc se vyhází a pro EeeAccount objekty bez odpovídajících
ESourceGroup objektů se ESourceGroup objekty vytvoří. Čili se nejedná o tupé
odstranění všech ESourceGroup a vytvoření seznamu kalendářů ``from the scratch".
Tímto se docílí minimálních problémů v GUI při periodických synchronizacích,
kdy většinu času není potřeba dělat v hierarchii ESourceGroup a ESource objektů
žádné změny.

\subsubsection{Offline režim}

V offline režimu neprobíhá synchronizace protože není možné vytvořit seznam
EeeAccount objektů.

To v jakém je evolution režimu se dá zjistit z tak, že pomocí pugin systému 
zaregistrujeme notifikaci při přechodu z jednoho režimu do druhého a v pluginu
si budeme uchovávat poslední stav.

Při přechodu z offline do online režimu se povolí periodické synchronizace
seznamu kalendářů a některé GUI prvky, které v offline režimu nejsou přístupné.
Zároveň se okamžitě provede resynchronizace seznamu kalendářů.

\subsubsection{Úprava kontextového menu}

Úpravy menu se provádí tak, že evolution při kliknutí na kalendář zavolá hook,
který může pozměnit konfiguraci menu.

Konfigurace menu obsahuje položky které jsou identifikované určitým klíčem, např. 20.delete
je tlačítko pro smazání kalendáře, 30.properties je tlačítko pro úpravu
vlastností kalendáře. My můžeme přidávat další položky a upravovat existující
(vyžaduje aplikaci patche).

Hook v 3E pluginu tedy zjistí zda bylo kliknuto na 3E kalendář a podle typu
kalendáře a online/offline stavu upraví konfiguraci menu.

Pokud uživatel vybere naši položku z menu, evolution zavolá námi nastavený
callback.

V offline režimu kontextové menu 3E kalendáře upravuje/přidává položky:
\begin{description}
\item [Configure ACL...] Zašedivěno.
\item [Unsubscribe] Zašedivěno.
\item [Delete] Zašedivěno.
\item [Properties...] Zašedivěno.
\end{description}

V online režimu kontextové menu 3E kalendáře vlastníka účtu upravuje/přidává položky:
\begin{description}
\item [Configure ACL...] Otevírá dialog pro nastavení práv kalendáře.
\item [Delete] Volá přímo z GUI pluginu RPC pro odstranění kalendáře a následně 
resynchronizuje seznam kalendářů.
\end{description}

\newpage
V online režimu kontextové menu 3E zapsaného kalendáře upravuje/přidává položky:
\begin{description}
\item [Unsubscribe] Volá přímo z GUI pluginu RPC pro
odhlášení kalendáře a následně resynchronizuje seznam kalendářů.
\item [Delete] Zašedivěno.
\end{description}

\subsubsection{Zápis kalendáře/úprava hlavního menu}

Pomocí konfiguračního souboru pluginu se přidá položka na určité místo v menu a
nastaví se callback který se zavolá při kliknutí.

Po kliknutí se otevře dialog pro zápis kalendáře.

\subsubsection{Vytvoření kalendáře a změna nastavení}

Toto se provádí pomocí standardního dialogu v evolution. Tento dialog obsahuje
widget GtkTable, který máme možnost upravovat pomocí plugin systému.

Rozšíření je implementováno pomocí tří callbacků:
\begin{description}
\item [properties\_factory] Tato funkce má za úkol sestavit/upravit GUI dialogu.
\item [properties\_check] V této funkci je možné zkontrolovat zda je možné
provést přidání/úpravu kalendáře. Pokud to není možné, tak je třeba vrátit FALSE
a tlačítko pro potvrzení změn bude zašedivěno.
\item [properties\_commit] Tato funkce se volá pro dokončení změn po stisknutí
tlačítka pro jejich potvrzení. Tj. v našem případě pro vytvoření kalendáře na 3E
serveru, resp. odeslání nového nastavení (barva, název kalendáře) na server.
Zde je třeba dát pozor na to že evolution sama přidá nový ESource do
ESourceList, jenže problém je že tento ESource je chybný z hlediska 3E
pluginu, takže je třeba po přidání provést okamžitě resynchronizaci.
\end{description}

V jeden moment může být aktivní pouze jeden dialog pro přidání, nebo úpravu nastavení
kalendáře. Funkce proto sdílejí data pomocí statických proměnných.\footnote{Nic
jiného nám ani nezbývá.}

Callbacky se vždy volají pro všechny skupiny kalendářů (New Calendar) a všechny
kalendáře (Properties...). Jiank řečeno, callbacky budou volány i pro standardní
typy kalendářů v Evolution (file, weather, atd.). Callback sám musí rozhodnout
zda je volán na správném typu kalendáře a má tedy provést výše uvedenou činnost
či ne. To se dá rozhodnout pomocí ESource obejktu kalendáře které je předáváno
callbacku. Typy kalendáře jsou rozlišované pomocí URI prefixu. 3E kalendáře mají
prefix eee://.

To zda se callbacky volají pro dialog pro přidání kalendáře, nebo pro dialog pro
úpravu nastavení kalednáře se určuje podle toho zda předávané ESource obsahuje
má nastavenu URI kalendáře.

\subsection{Zápis sdíleného kalendáře}

Zápis sdíleného kalendáře je implementován jedním dialogem, který obsahuje
quiksearch řádek pro filtrování seznamu kalendářů podle uživatele a tabulku se
seznamem sdílených kalendářů. Uživatel si vybere kalendář z tabulky a klikne na
``Apply".

GUI dialogu je napsané v glade, konstruktor dialogu sestaví GUI z glade souboru
a zaregistruje callbacky. Každý callback dostane v nepoviném parametru
\textit{data} objekt obsahující stav dialogu (\textit{struct subscribe\_context}).

Díky tomuto objektu každý callback ví ``co se děje", tj. v jakém stavu se celý
dialog nachází.

ACL dialog ignoruje vlastní účet uživatele a již zapsané kalendáře.

Po kliknutí na ``Apply" se klient připojí k serveru přes jeden z účtů uživatele
a provede zapsání kalednáře. Nakonec se provede resynchronizace seznamu
kalednářů.

\subsection{Nastavení ACL}

Nastavení ACL je implementováno podobně jako zápis sdíleného kalendáře jedním
dialogem a kontextovým objektem (\textit{struct acl\_context}).

Po otevření dialogu se načte seznam práv pro vybraný kalendář a z tohoto seznamu
se pak určí režim přístupu ke kalendáři:
\begin{description}
\item [Private] Seznam práv je prázdný.
\item [Public] Seznam práv obsahuje divokou kartu '*'.
\item [Shared] Seznam pr8v obsahuje alespoň jednoho uživatele s právy read nebo
write.
\end{description}

Uživatel pak může pomocí dialogu nastavit režim přístupu ke kalendáři a případně
určit seznam uživatelů a jejich práva.

Při potvrzení změn se porovná aktuální nastavení a nastavení na serveru a
provedou se potřebné změny.

\newpage
\section{EDS plugin}

\subsection{Vlastnosti}

Potřebujeme synchronizovat obsah kalendáře mezi několika CUA a jedním serverem.

Změny v kalendáři, které mohou nastat jsou:
\begin{itemize}
\item Přidání nového objektu,
\item Odstranění objektu,
\item Úprava existujícího objektu.
\end{itemize}

Potřebujeme také umožnit omezenou možnost práce s kalendářem v offline režimu.
Změny mohou vzniknout nas serveru bez vědomíi CUA (tj. CUA neprovedl dané změny).
Cílem je aby všichni klienti pracující v online režimu viděli aktuální stav
kalendáře tak jak je uložený na serveru.

\subsection{Poznámky k implemetaci}

EDS plugin implementuje přístup k obsahu jednoho kalendáře. Je tvořen jedním
objektem jehož metody se dají rozdělit do těchto skupin:
\begin{description}
\item[Manipulace s objekty] Přidání, odstranění, aktualizace objektu, vyhledání,
získání objektu. Tyto operace neprovádějí přímo komunikaci se serverem, ale
pouze operují nad lokálním mirrorem obsahu kalendáře na serveru.
\item[Maniplace s kalendářem] Otevření, vytvoření a zrušení kalendáře. Tyto
metody se nevyužívají k jejich původnímu účelu, protože manipulace s kalendáři
se provádí z Evolution pluginu.
\item[Zasílání a příjem iTipů] Zasílání a příjem iTipů přes 3E protokol.
\item[Utility metody] Získání implementovaných vlastnosí backendu, získání
emailu uživatele, nastavení alarmu, atd.
\end{description}

\subsubsection{Synchronizace objektů v kalendáři}

Metody pro manipulaci budou operovat nad lokální keší kalendáře, která se bude
pravidleně oboustranně synchronizovat se serverem. Rozhraní cache umožňuje
vynutit synchronizaci jednoho konkrétního objektu. (V online režimu, aby si
metody backendu mohly vynutit okamžitouo aktualizaci kalendáře.)

Synchronizace má dva směry:
\begin{description}
\item[CUA to Server] V lokální cache v CUA se budou změněné objekty ozančovat
pomocí vlastnosti X-EEE-STATUS, která může nabývat hodnot ADDED, MODIFIED,
REMOVED. Keš se bude periodicky procházet a všechny změny u označených objektů
se provedou na serveru a ozančení se zruší. Synchronizaci je možné vynutit
pro jednu konkrétní změnu. To slouží k okamžitému provádění změn.
\item[Server to CUA] EDS plugin periodicky kontaktuje server a získáva z něj
objekty, které byly změněny od poslední aktualizace.
\end{description}

\subsubsection{Offline režim}

V offline režimu backend plugin nekomunikuje se serverem. Pracuje se pouze nad
lokální keší objektů. Při přechodu z offline do online režimu se provede
synchronizace keše.

\section{Úpravy Evolution a EDS}

Seznam ůprav potřebných pro správnou funkci pluginů:
\begin{description}
\item[Export hlavičkových souborů] Evolution při instalaci neinstaluje množství
potřebných hlavičkových souborů, které jsou součástí plugin systému, takže
externí pluginy nemohou využívat všech možností. Tyto hlavičkové soubory lze
instalovat buď úpravou Makefile souborů, nebo pomocí sestavovacího skriptu pro
binární balíček po sestavení samotné evolution. Zvolil jsem druhé řešení,
protože nevyžaduje patchování evolution.
\item[Přetěžování položek v kontextovém menu] Toto chování je zdokumentované,
ale není naimplementované. Můj patch toto chování implementuje.
\item[Signalizace aktivace komponenty] Implementace generování události
v rámci plugin systému evolution při aktivaci komponenty.
\item[Autentizace skupiny kalendářů] EDS identifikuje hesla v klíčence podle URI
kalendáře. Díky tomu není možné mít v klíčence jedno heslo pro skupinu kalendářů
přidružených k jednomu účtu. Tento patch implemntuje možnost nastavení auth-key
u ESource objektu, pro ruční nastavení klíče odlišného od URI kalendáře.
\end{description}

Tyto patche jsou k dispozici v zdrojovém balíku 3E pluginů.

\section{Poznámky}

Nejsou.

\end{multicols*}

\end{document}
